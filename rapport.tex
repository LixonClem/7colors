\documentclass[a4paper,11pt]{article}

%%%%%%%%%%%%%%%%%%%%%%%%%%%%%%%%%%%%%%%%%%%%%%%%%%%%%%%%%%%%

% Global structure parameters
\usepackage{fullpage}%

\usepackage[francais]{babel}%

\usepackage[utf8]{inputenc}%
\usepackage[T1]{fontenc}%

% Font selection: http://www.tug.dk/FontCatalogue/newpx/
\usepackage{newpxtext}%
\usepackage{newpxmath}%

% Macro packages
\usepackage{url}%
\usepackage{graphicx}%
\usepackage{listings}%

% Parameters for listings
\lstset{%
  basicstyle=\footnotesize\sffamily,%
  columns=fullflexible,%
  frame=lb,%
  frameround=fftf,%
  language=caml,%
}%

% Fine tuning
\setlength{\parskip}{0.2\baselineskip plus 0.2\baselineskip}%

%%%%%%%%%%%%%%%%%%%%%%%%%%%%%%%%%%%%%%%%%%%%%%%%%%%%%%%%%%%%

\begin{document}

\title{Rapport du projet 7 couleurs}

\author{Le Dilavrec Quentin et Clément Legrand Lixon}

\date{15 octobre 2017}

\maketitle

\begin{abstract}
Le jeu des 7 couleurs est un jeu vidéo de stratégie.
Que nous avons réimplémenté en C. Ce rapport à pour but de répondre 
clairement et de façon détaillé à chacunes des questions  que nous avons
traitées présentes dans le sujet du projet.
\end{abstract}

%%%%%%%%%%%%%%%%%%%%%%%%%%%%%%%%%%%%%%%%%%%%%%%%%%%%%%%%%

\section{Introduction}
Le jeu des 7 couleurs aussi applé Filler dans sa version anglophone est 
un jeu video de strategie/puzzle créé par Dmitry Pashkov et pour la première 
fois publié par la société Gamos pour MS-DOS en 1990! De par son gameplay 
très simple , peu d'intéractions avec l'utilisateur tant du point de vue des commandes 
joué par le(s) utilisateur(s) que par l'affichage facilement faisable dans le shell. 
Le jeu est donc parfaitement adapté au programmeurs débutants en C. Le sujet est 
pésenté sous forme de questions chacunes portant sur une partie particulière du jeu, 
de l'affichage à l'interation avec l'utilisateur pour finir avec les ia.

%%%%%%%%%%%%%%%%%%%%%%%%%%%%%%%%%%%%%%%%%%%%%%%%%%%%%%%%%

\section{L'affichage}

Dans cette première partie, nous nous sommes intéressés à l'affichage
du monde des 7 couleurs, ainsi que sa mise à jour après un coup d'un des
joueurs.

\subsection{Initialisation du plateau de jeu}
\emph{Reponse de Q1}

Pour permettre l'initialisation du plateau de jeu, il faut choisir de manière
aléatoire l'une des 7 couleurs disponibles, représentées par la suite par 
une lettre (A, B, C, ..., G). Il suffit alors de tirer un entier aléatoire entre 1 et 7
puis de le remplacer par la lettre correspondante dans le plateu de jeu. 

Pour cela nous avions dans un premier temps  utilisé la fonction rand() 
disponible en C, qui renvoie un entier, puis on calculait le reste de sa division 
par 7 pour obtenir un entier entre 0 et 6. 

Toutefois cette méthode posait un problème puisque nous obtenions le même plateau 
de jeu à chaque nouvelle partie. Pour corriger ce problème nous définissons
la variable RANDOM COLOR de la manière suivante : \#define RANDOM COLOR 1 + (int)(rand() \% 7).
De cette manière c'est toujours une fonction rand() différente qui est utilisée et 
l'entier est obtenu plus aléatoirement qu'avant, ce qui permet d'avoir un tableau
de jeu différent à chaque partie.

De plus il ne faut pas remplir les cases (29,0) et (0,29) qui contiennent 
respectivement les symboles des joueurs 1 et 2. A l'issue de cette initialisation, 
nous pouvons nous intéresser au déroulementdu jeu en lui-même.

\subsection{Mise à jour du plateau}
\emph{Reponse de Q2}

Pour pouvoir (enfin) jouer au jeu, il faut que le plateau se mette à jour
automatiquement une fois que l'un des joueurs a joué une lettre. Pour ce faire
on parcourt le monde linéairement jusqu'à tomber sur une lettre à la fois choisie et
voisine d'un symbole joueur. Dans ce cas on remplace la lettre trouvée par un symbole
joueur et on recommence le parcours du monde du début, de cette façon on est sûr de 
n'oublier aucune lettre lors de la mise à jour, et l'algorithme se termine bien 
puisqu'il n'y a qu'un nombre fini de cases dans le plateau de jeu. Pour vérifier
que cette fonction affiche le résultat voulu, il suffit de la tester sur un plateau, 
de rentrer une lettre et de vérifier qu'il ne reste plus de lettres choisies adjacentes
à un symbole joueur. Le pire cas est obtenu lorsque le plateau ne contient qu'une seule
lettre et qu'elle est choisie par un joueur : le monde est alors parcouru autant de fois
qu'il y a de cases comportant une lettre c'est-à-dire 898 fois (900 moins les deux cases
joueurs initiales). 

%%%%%%%%%%%%%%%%%%%%%%%%%%%%%%%%%%%%%%%%%%%%%%%%%%%%%%%%%


\section{Les interactions avec l'utilisateur}

A présent nous allons développer une implémentation Humain/Humain, pour 
pouvoir jouer au jeu 7colors à deux.

\subsection{De quoi jouer faire un duel entre utilisateurs}
\emph{Reponse de Q4}

Pour que deux joueurs humains puissent s'affronter dans ce jeu, il faut d'abord
s'assurer que chaque joueur puisse jouer l'un après l'autre. Il suffit pour cela 
de choisir de manière arbitraire quel joueur commencera la partie (il reçoit la
valeur "true" puis une fois que son tour est terminé, il prend la valeur "false" et
cette fois c'est au tour de l'autre joueur. Il suffit ensuite de demander au joueur
une lettre à remplacer pour agrandir sa zone, puis de mettre à jour le plateau de jeu.
Toutefois il peut aussi arriver que l'un des deux joueurs tape (de manière tout à 
fait malencontreuse) un caractère différent de l'une des lettres attendues 
(c'est-à-dire une lettre majuscule (ou minuscule) entre A et G), et dans ce cas on
doit renvoyer un message d'erreur adapté pour que le joueur puisse corriger. 
Cette implémentation est limitée car ?
Comme aucune condition de victoire n'a été définie, la partie entre les deux joueurs
est sans fin... Il est temps d'y remédier !  


\subsection{Conditions de victoire}
\emph{Reponse de Q5}

Une première idée serait d'arrêter la partie seulement lorsque le monde est
divisée en deux zones, puis de comparer le nombre de cases de chacun des joueurs.
Toutefois, dès que l'un des deux joueurs possède au moins 50\% du plateau, l'autre joueur
ne pourra plus gagner : c'est notre condition de victoire. 
Ainsi la partie s'arrête dès que l'un des deux joueurs possède au moins 50\% du plateau.
On calcule donc à chaque tour le nombre de cases possédées par chaque joueur, on en
déduit le pourcentage d'occupation du plateau par chaque joueur (ce que l'on affiche
à chaque tour), et on peut alors savoir si la condition de victoire est réalisée pour
l'un des deux joueurs.

%%%%%%%%%%%%%%%%%%%%%%%%%%%%%%%%%%%%%%%%%%%%%%%%%%%%%%%%%

\section{Les intelligences artificielles aléatoires}

C'est bien de pouvoir jouer à deux, mais on n'a pas toujours d'amis sous la
main, et pour pallier à ce problème nous allons nous intéresser à présent 
à l'implémentation de certaines Intelligences artificielles.

\subsection{Jouer totalement aléatoirement}

La première IA à programmer doit choisir aléatoirement une couleur parmi 
les 7 du jeu, lorsque c'est son tour. Pour ce faire il suffit de reprendre 
l'implémentation pour un joueur humain, sauf qu'au lieu de rentrer une couleur
soi-même, elle est choisie de manière aléatoire (comme quand on rempli le tableau
pour la première fois (cf Question 1)). Le plateau est ensuite mis à jour 
et affiché. On demande également à ce que la couleur choisie soit affichée.
On peut observer, après plusieurs parties, que cette IA est particulièrement 
inefficace, on va donc s'intéresser à une IA aléatoire plus efficace.


\subsection{Jouer aléatoirement plus intelligemment}
\emph{Reponse de Q7}

Cette IA, a pour objectif d'être plus efficace que l'IA précédente. On va donc
s'intéresser cette à fois une IA qui choisi une couleur aléatoirment mais 
seulement parmi les couleurs qui peuvent agrandir sa zone. 

Pour connaître les couleurs adjacentes à l'IA, on initialise un tableau de
booléens B avec des "false", puis on parcourt le plateau jusqu'à tomber sur une lettre qui possède 
un voisin correspondant au symbole de l'IA, on transforme ensuite la lettre 
en un chiffre n puis le n-ième élément de B prend la valeur "true", puis on continue
le parcours du plateau. A la fin on obtient un tableau B qui permet de connaître
les couleurs adjacentes à l'IA. Il suffit enfin de choisir un entier aléatoire entre 
0 et 6 qui renvoie vers une valeur "true" de B, puis de transformer cet entier
en sa couleur associée. On peut alors mettre à jour le plateau de jeu, et afficher la
couleur choisie.

Cette IA est nettement plus efficace que la précédente mais peut-on faire mieux ?

%%%%%%%%%%%%%%%%%%%%%%%%%%%%%%%%%%%%%%%%%%%%%%%%%%%%%%%%%

\section{L' intelligence artificielle gloutonne}

La partie précédente nous a permis de développer des intelligences artificielles
basées sur l'aléatoire, mais l'aléatoire a souvent des limites. Une idée naïve pour
qu'une IA gagne serait qu'à chaque tour elle essaie de récupérer le maximum de cases
autour d'elle. L'objectif de cette partie est donc d'implémenter cette IA, qualifiée 
de gloutonne, puis de la comparer à l'IA aléatoire implémentée précédemment.


\subsection{Maximisation du gain en cases immédiate}
\emph{Reponse de Q8}

Pour implémenter cette nouvelle IA, on construit une copie vierge du plateau de jeu,
ainsi qu'un tableau d'entiers T, dont chaque case contiendra le nombre de cases que l'IA peut
avoir en jouant la lettre associée. 

On commence par parcourir le plateau de jeu, lorsqu'on tombe sur une lettre voisine
d'un symbole joueur, ou voisine de la même lettre dans le plateau copie, on place cette
lettre dans le plateau copie à la même place que dans le vrai plateau, on incrémente
la i-ème case de T de 1 (avec i le nombre associé à la lettre trouvée) et on 
recommence le parcours du plateau du début. A la fin, T contient le nombre de cases
que l'IA pourra récupérer en jouant une lettre, il suffit alors de récupérer l'indice
du maximum de T, puis l'IA jouera la couleur associée à cet indice. 

Nous obtenons ainsi une IA qui cherche à maximiser son nombre de cases à chaque tour,
essayons maintenant de faire jouer notre IA aléatoire améliorée et cette IA, pour
savoir laquelle est la meilleure.

\subsection{Aléatoire contre Glouton}
\emph{Reponse de Q9}



\subsection{Championnat}
\emph{Reponse de Q10}

aaa

\section{Conclusion}

aaaaaaaaaaaaaaaaaaa

\end{document}
