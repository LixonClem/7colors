\documentclass[a4paper,11pt]{article}

%%%%%%%%%%%%%%%%%%%%%%%%%%%%%%%%%%%%%%%%%%%%%%%%%%%%%%%%%%%%

% Global structure parameters
\usepackage{fullpage}%

\usepackage[francais]{babel}%

\usepackage[utf8]{inputenc}%
\usepackage[T1]{fontenc}%

% Font selection: http://www.tug.dk/FontCatalogue/newpx/
\usepackage{newpxtext}%
\usepackage{newpxmath}%

% Macro packages
\usepackage{url}%
\usepackage{graphicx}%
\usepackage{listings}%

% Parameters for listings
\lstset{%
  basicstyle=\footnotesize\sffamily,%
  columns=fullflexible,%
  frame=lb,%
  frameround=fftf,%
  language=caml,%
}%

% Fine tuning
\setlength{\parskip}{0.2\baselineskip plus 0.2\baselineskip}%

%%%%%%%%%%%%%%%%%%%%%%%%%%%%%%%%%%%%%%%%%%%%%%%%%%%%%%%%%%%%

\begin{document}

\title{Rapport du projet 7 couleurs}

\author{Le Dilavrec Quentin et Clément Legrand Lixon}

\date{15 octobre 2017}

\maketitle

\begin{abstract}
Le jeu des 7 couleurs est un jeu vidéo de stratégie.
Que nous avons réimplémenté en C. Ce rapport à pour but de répondre clairement et de façon détaillé à chacunes des questions  que nous avons traitées présentent dans le sujet du projet.
\end{abstract}

%%%%%%%%%%%%%%%%%%%%%%%%%%%%%%%%%%%%%%%%%%%%%%%%%%%%%%%%%

\section{Introduction}
Le jeu des 7 couleurs aussi applé Filler dans sa version anglophone est un jeu video de strategie/puzzle créé par Dmitry Pashkov et pour la première fois publié par la société Gamos pour MS-DOS en 1990! De par son gameplay très simple , peu d'intérations avec l'utilisateur tant du point des commandes joué par le(s) utilisateur(s) que par l'affichage facilement faisable dans le shell. Le jeu est donc parfaitement adapté au programmeurs débutants en C. Le sujet est pésenté sous forme de questions chacunes portant sur une partie particulière du jeu, de l'affichage à l'interation avec l'utilisateur pour finir avec les ia.

%%%%%%%%%%%%%%%%%%%%%%%%%%%%%%%%%%%%%%%%%%%%%%%%%%%%%%%%%

\section{L'affichage}

aaaaaaaaaaaaaaaaaaaaaaaaaaaa

\subsection{Initialisation du plateau de jeu}
\emph{Reponse de Q1}

aaa

\subsection{Mise à jour du plateau}
\emph{Reponse de Q2}

aaa


%%%%%%%%%%%%%%%%%%%%%%%%%%%%%%%%%%%%%%%%%%%%%%%%%%%%%%%%%


\section{Les interaction avec l'utilisateur}

aaaaaaaaaaaaaaaaaaaaaaaaaaaa


\subsection{De quoi jouer faire un duel entre utilisateurs}
\emph{Reponse de Q4}

aaa

\subsection{Conditions de victoire}
\emph{Reponse de Q5}

aaa

%%%%%%%%%%%%%%%%%%%%%%%%%%%%%%%%%%%%%%%%%%%%%%%%%%%%%%%%%

\section{Les inteligences artificielle aléatoires}

aaaaaaaaaaaaaaaaaaaaaaaaaaaa


\subsection{Jouer totalement aléatoirement}
\emph{Reponse de Q6}

aaa

\subsection{Jouer aléatoirement plus intelligemment}
\emph{Reponse de Q7}

aaa

%%%%%%%%%%%%%%%%%%%%%%%%%%%%%%%%%%%%%%%%%%%%%%%%%%%%%%%%%

\section{Les inteligences artificielle gloutonne}

aaaaaaaaaaaaaaaaaaaaaaaaaaaa


\subsection{Maximisation du gain en cases immédiat}
\emph{Reponse de Q8}

aaa

\subsection{Aléatoire contre Glouton}
\emph{Reponse de Q9}

aaa

\subsection{Championat}
\emph{Reponse de Q10}

aaa

\section{Conclusion}

aaaaaaaaaaaaaaaaaaa

\end{document}
